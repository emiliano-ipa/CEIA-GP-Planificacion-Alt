\documentclass[
11pt, % The default document font size, options: 10pt, 11pt, 12pt
%codirector, % Uncomment to add a codirector to the title page
]{charter} 

% Preambulo
\usepackage{tabularx}
\usepackage[table,xcdraw]{xcolor}

% El títulos de la memoria, se usa en la carátula y se puede usar el cualquier lugar del documento con el comando \ttitle
\titulo{Aplicación de inteligencia artificial a la planificación financiera de productos secos en el ciclo III: demanda, márgenes y simulación de escenarios} 

% Nombre del posgrado, se usa en la carátula y se puede usar el cualquier lugar del documento con el comando \degreename
%\posgrado{Carrera de Especialización en Sistemas Embebidos} 
%\posgrado{Carrera de Especialización en Internet de las Cosas} 
\posgrado{Carrera de Especialización en Inteligencia Artificial}
%\posgrado{Maestría en Sistemas Embebidos} 
%\posgrado{Maestría en Internet de las cosas}

% Tu nombre, se puede usar el cualquier lugar del documento con el comando \authorname
% IMPORTANTE: no omitir titulaciones ni tildación en los nombres, también se recomienda escribir los nombres completos (tal cual los tienen en su documento)
\autor{Mg. Emiliano Daniel Iparraguirre}

% El nombre del director y co-director, se puede usar el cualquier lugar del documento con el comando \supname y \cosupname y \pertesupname y \pertecosupname
\director{Título y Nombre del director}
\pertenenciaDirector{pertenencia} 
\codirector{} % para que aparezca en la portada se debe descomentar la opción codirector en los parámetros de documentclass
\pertenenciaCoDirector{FIUBA}

% Nombre del cliente, quien va a aprobar los resultados del proyecto, se puede usar con el comando \clientename y \empclientename
\cliente{Santiago Pagola}
\empresaCliente{Cagnoli S.A.}
 
\fechaINICIO{29 de abril de 2025}		%Fecha de inicio de la cursada de GdP \fechaInicioName
\fechaFINALPlan{17 de junio de 2025} 	%Fecha de final de cursada de GdP
\fechaFINALTrabajo{xx de xxx de 2026}	%Fecha de defensa pública del trabajo final


\begin{document}

\maketitle
\thispagestyle{empty}
\pagebreak


\thispagestyle{empty}
{\setlength{\parskip}{0pt}
\tableofcontents{}
}
\pagebreak


\section*{Registros de cambios}
\label{sec:registro}


\begin{table}[ht]
\label{tab:registro}
\centering
\begin{tabularx}{\linewidth}{@{}|c|X|c|@{}}
\hline
\rowcolor[HTML]{C0C0C0} 
Revisión & \multicolumn{1}{c|}{\cellcolor[HTML]{C0C0C0}Detalles de los cambios realizados} & Fecha      \\ \hline
0      & Creación del documento                                 &\fechaInicioName \\ \hline
1      & Se completa hasta el punto 5 inclusive                & {13} de {mayo} de 2025 \\ \hline
2      & Se completa hasta el punto 9 inclusive                & {20} de {mayo} de 2025 \\ \hline
3      & Se completa hasta el punto 12 inclusive                & {27} de {mayo} de 2025 \\ \hline
4      & Se completa el plan	                                 & {3} de {junio} de 2025 \\ \hline
5      & Se realizan las correcciones finales	                  & {16} de {junio} de 2025 \\ \hline

% Si hay más correcciones pasada la versión 4 también se deben especificar acá

\end{tabularx}
\end{table}

\pagebreak



\section*{Acta de constitución del proyecto}
\label{sec:acta}

\begin{flushright}
Buenos Aires, \fechaInicioName
\end{flushright}

\vspace{2cm}

Por medio de la presente se acuerda con el \authorname\hspace{1px} que su Trabajo Final de la \degreename\hspace{1px} se titulará ``\ttitle'' y consistirá en \textcolor{black}{desarrollar un modelo basado en inteligencia artificial para predecir la demanda y los márgenes brutos de productos secos en Cagnoli S.A., integrándolo al proceso de planificación financiera para mejorar la toma de decisiones y simular escenarios}. El trabajo tendrá un presupuesto preliminar estimado de 677 horas y un costo de USD 27.080, con fecha de inicio el \fechaInicioName\hspace{1px} y fecha de presentación pública el \textcolor{red} {\fechaFinalName}.

Se adjunta a esta acta la planificación inicial.

\vfill

% Esta parte se construye sola con la información que hayan cargado en el preámbulo del documento y no debe modificarla
\begin{table}[ht]
\centering
\begin{tabular}{ccc}
\begin{tabular}[c]{@{}c@{}}Dr. Ing. Ariel Lutenberg \\ Director posgrado FIUBA\end{tabular} & \hspace{2cm} & \begin{tabular}[c]{@{}c@{}}\clientename \\ \empclientename \end{tabular} \vspace{2.5cm} \\ 
\multicolumn{3}{c}{\begin{tabular}[c]{@{}c@{}} \supname \\ Director del Trabajo Final\end{tabular}} \vspace{2.5cm} \\
\end{tabular}
\end{table}




\section{1. Descripción técnica-conceptual del proyecto a realizar}
\label{sec:descripcion}

El presente proyecto se realiza en el marco de la Especialización en Inteligencia Artificial, y surge a partir de la experiencia profesional del alumno como consultor en planificación financiera e inteligencia de negocios en Cagnoli S.A., una empresa argentina con más de un siglo de trayectoria en la industria de alimentos. La firma se especializa en la elaboración de fiambres y embutidos con estándares de calidad certificados, y cuenta con distintas unidades productivas organizadas en ciclos. En particular, este trabajo se enfoca en el \textbf{Ciclo III} (producción de fiambres y chacinados), correspondiente a la producción de productos secos (salamines), una de las líneas más relevantes por su imagen de marca, volumen de ventas y contribución al margen.

El objetivo del proyecto es diseñar una herramienta de planificación financiera basada en Inteligencia Artificial, orientada a pronosticar la demanda y estimar márgenes brutos por producto y canal. Además, se desarrollará un módulo de simulación de escenarios operativos y financieros que permita evaluar el impacto ante posibles variaciones en factores críticos como mermas productivas, precios de insumos, regulaciones o cambios en el patrón de consumo. Esta herramienta busca integrarse al proceso de planificación mensual que coordina las áreas Comercial, Producción y Finanzas, con el fin de mejorar así la calidad de las decisiones.

No existen actualmente acuerdos de confidencialidad o financiamiento externo. Los entregables serán propiedad del alumno, y se resguardará la confidencialidad comercial en caso de futuras publicaciones académicas.

\subsection*{Contexto operativo}
El proceso de elaboración de productos secos implica tiempos prolongados de maduración y secado (que pueden superar los 30 días), por lo que una planificación incorrecta genera consecuencias graves: desabastecimiento, vencimientos, capacidad ociosa o pérdida de clientes. Además, se enfrenta una alta estacionalidad: la demanda se concentra en verano, con marcada caída en los meses fríos. 

La empresa, si bien ha incorporado recientemente herramientas de control de gestión y planificación, aún no cuenta con modelos predictivos robustos que integren física y financieramente el negocio.

\subsection*{Aspectos técnicos y estado del arte}
La solución se encuentra en una etapa temprana del ciclo de vida: diseño conceptual, análisis del problema y planificación del desarrollo. Durante el proyecto se avanzará hasta el desarrollo funcional del modelo, su evaluación en entorno de pruebas y su validación con datos reales, sin incluir en el alcance aún la integración de la solución con los sistemas transaccionales de la empresa.

En la figura \ref{fig:diagBloques} se presenta el diagrama en bloques del sistema propuesto. Se observa cómo, a partir de diversas fuentes de datos (ventas históricas, costos variables, eventos comerciales y variables externas), el sistema procesa la información mediante tres módulos: uno de predicción de demanda (basado en SARIMA, Prophet y algoritmos de \textit{machine learning} supervisado como XGBoost y Random Forest), uno de estimación de márgenes (que calcula costos y contribución marginal proyectada), y otro de simulación de escenarios (utilizando árboles de decisión y análisis "what if"). La salida de este procesamiento se visualiza mediante Power BI o Streamlit, generando reportes que sirven como insumo clave para las reuniones mensuales de planificación operativa y financiera.

\vspace{15px}

\begin{figure}[htpb]
\centering 
\includegraphics[width=.65\textwidth]{./Figuras/Diagrama_proyecto.png}
\caption{Diagrama en bloques del sistema.}
\label{fig:diagBloques}
\end{figure}

\vspace{25px}


\subsection*{Cliente y valor esperado}
El principal cliente del proyecto es el equipo de planificación, ya que requiere información confiable y anticipada para mejorar sus decisiones. La innovación del proyecto radica en conectar el planeamiento operativo y financiero con modelos predictivos y simulación en una plataforma visual y accesible.
La solución permitirá analizar semanalmente las proyecciones de demanda y márgenes por canal, junto con simulaciones de impacto financiero ante cambios en precios de insumos o contexto de mercado. Será utilizada como insumo objetivo en las reuniones mensuales de planificación comercial y presupuestaria.

\section{2. Identificación y análisis de los interesados}
\label{sec:interesados}


\begin{table}[ht]
%\caption{Identificación de los interesados}
%\label{tab:interesados}
\begin{tabularx}{\linewidth}{@{}|l|X|X|l|@{}}
\hline
\rowcolor[HTML]{C0C0C0} 
Rol           & Nombre y Apellido & Organización 	& Puesto 	\\ \hline
Cliente       & Santiago Pagola      &	Cagnoli S.A.	&	Responsable de Planificación       	\\ \hline
Impulsor      & Juan Pedro Cagnoli                  &  Cagnoli S.A.            	&  Gerente de Operaciones      	\\ \hline
Responsable   & Emiliano Iparraguirre       & FIUBA        	& Alumno 	\\ \hline
Orientador    & \supname	      & \pertesupname 	& Director del Trabajo Final \\ \hline
Usuario final & Analistas de Planificación                  & Cagnoli S.A.             	&  -      	\\ \hline
\end{tabularx}
\end{table}
\pagebreak

A continuación, se listan las principales características de cada interesado:
\begin{itemize}
	\item Cliente: el Cr. Santiago Pagola colaborará con la explicación del proceso de planificación, la definición de los requerimientos y la coordinación de ejecución de pruebas de aceptación.
	\item Impulsor: el Ing. Juan Pedro Cagnoli es el principal referente y sponsor de este proyecto dentro de la organización. Esta acción se inserta en un conjunto de iniciativas estratégicas que él mismo ha impulsado, con el propósito de perfeccionar la planificación, aumentar la precisión de los pronósticos y sustentar las decisiones empresariales en modelos objetivos.
	\item Orientador: \textcolor{red}{características del orientador.}
	\item Usuario final: los analistas del Departamento de Planificación son los usuarios finales de esta herramienta, antes y durante las reuniones de planificación, junto al resto de las áreas involucradas (Comercial, Producción, Logística y Compras).
\end{itemize}


\section{3. Propósito del proyecto}
\label{sec:proposito}

El propósito de este proyecto es desarrollar un modelo predictivo basado en inteligencia artificial que prevea la demanda y los márgenes brutos por producto y canal, integrando datos del sistema ERP de la compañía, costos dinámicos e información comercial. Adicionalmente, como último paso, se incluirá una capa de simulación de escenarios para evaluar impactos de cambios en variables críticas sobre los resultados financieros. Esta solución permite aportar herramientas objetivas para la planificación operativa y financiera integrada en un flujo de trabajo inteligente, proporcionando información confiable y anticipada para la toma de decisiones.


\section{4. Alcance del proyecto}
\label{sec:alcance}

El proyecto se centrará exclusivamente en la familia de productos secos (salamines), por su peso estratégico dentro del portafolio de Cagnoli S.A. Estos productos requieren una planificación diferenciada debido a sus procesos largos de secado y maduración, y a la alta estacionalidad en su demanda.


El presente proyecto incluye:
\begin{itemize}
	\item El diseño y validación de modelos de predicción de demanda, basados en algoritmos de series temporales (SARIMA, Prophet) y métodos de aprendizaje automático supervisado (XGBoost, Random Forest).
	\item La estimación proyectada de márgenes brutos por canal, a partir de precios previstos y costos dinámicos asociados a producción y distribución.
	\item La simulación de escenarios mediante árboles de decisión y análisis de sensibilidad, para evaluar impactos financieros ante variaciones en precios, demanda o restricciones operativas.
	\item Una prueba de concepto visual (PoC), implementada en Power BI y/o Streamlit, destinada a facilitar la interpretación y evaluación de los resultados por parte de los equipos internos.

\end{itemize}
\pagebreak

El presente proyecto no incluye:
\begin{itemize}
	\item El entrenamiento formal a usuarios finales o su incorporación en la rutina operativa de la empresa.
	\item La integración técnica con sistemas transaccionales o con el sistema ERP de la compañía (Microsoft Dynamics).
	\item La incorporación de otras líneas de productos como cocidos o frescos al análisis, que se encuentran más allá de la familia de productos secos.

\end{itemize}


\section{5. Supuestos del proyecto}
\label{sec:supuestos}

Para el desarrollo del presente proyecto se supone que:

\begin{itemize}
	\item Se contará con acceso oportuno y completo a los \textit{datasets} históricos necesarios para el modelado: ventas, costos, inventarios, estacionalidad, calendario comercial, entre otros. Los datos estarán en formato utilizable y reflejarán fielmente la realidad operativa del negocio.
	\item Se prevé que tanto el alumno como los referentes internos de la empresa (Departamento de Planificación) podrán destinar el tiempo necesario para reuniones, revisiones de avances, validación de entregables y provisión de información contextual.
	\item Se considera que las herramientas previstas (Python, bibliotecas de IA, Streamlit, Power BI, etc.) son adecuadas para el desarrollo del sistema, y que no surgirán restricciones técnicas críticas que impidan ejecutar los modelos o visualizar los resultados.
	\item El desarrollo se llevará a cabo con los recursos disponibles del alumno (hardware personal, conectividad y acceso a herramientas de código abierto). No se prevé la necesidad de inversiones adicionales en infraestructura o licencias.
	
	\item Se asume que no existirán bloqueos internos, resistencias de parte de otras áreas o barreras políticas que interfieran con el acceso a la información o con la validación de resultados por parte del equipo de planificación.
	\item Para las simulaciones y la interpretación de resultados, se asume que no ocurrirán cambios abruptos en el contexto macroeconómico (inflación, precios regulados, normativas sanitarias) que invaliden los patrones históricos o generen disrupciones graves en la operatoria de la empresa.
	\item Se considera que los datos históricos reflejan adecuadamente la variabilidad esperada del negocio, y que son válidos para el entrenamiento y evaluación de modelos predictivos.
\end{itemize}
\newpage

\section{6. Product Backlog}
\label{sec:backlog}

Los criterios para calcular los story points estarán divididos de la siguiente manera:
\begin{itemize}
	\item \textbf{Dificultad (1–5)}: evalúa nivel de esfuerzo técnico y de desarrollo requerido.
	\item \textbf{Complejidad (1–5)}: referido al grado de conocimiento y experiencia necesarios.
	\item \textbf{Riesgo (1–5)}: pondera la incertidumbre técnica, disponibilidad de datos, o dependencias externas.
\end{itemize}

Para la ponderación en cada uno de ellos se utilizará la serie de Fibonacci desde el 1 al 13. La escala queda de la siguiente forma: muy bajo (1), bajo (3), medio (5), alto (8) y muy alto (13).

\subsection*{6.1. Épica 1: modelado de demanda}
\begin{itemize}
  \item \textbf{Historia de usuario 1:} como analista de planificación operativa, quiero contar con un modelo de predicción de demanda semanal por producto, para anticipar las necesidades de producción.
    \begin{itemize}
      \item \textbf{Dificultad}: medio (5). Requiere diseño, entrenamiento y validación de modelos.
      \item \textbf{Complejidad}: medio (5). Involucra series temporales con estacionalidad marcada y ruido.
      \item \textbf{Riesgo}: bajo (3). Depende de la calidad y granularidad de los datos históricos.
      \item \textbf{Story Points} = 5 + 5 + 3 = 13.
      \item \textbf{Prioridad}: alta.
    \end{itemize}
  \item \textbf{Historia de usuario 2:} como responsable comercial, quiero contar con alertas ante desvíos significativos entre demanda proyectada y real para poder ajustar rápidamente la estrategia de ventas.
    \begin{itemize}
      \item \textbf{Dificultad}: bajo (3). Requiere definir umbrales y condiciones de alerta.
      \item \textbf{Complejidad}: bajo (3). Necesita cálculos comparativos en tiempo real.
      \item \textbf{Riesgo}: muy bajo (1). El riesgo se acota al mal ajuste de los parámetros de alerta.
      \item \textbf{Story Points} = 3 + 3 + 1 = 8 (valor próximo en la serie de Fibonacci).
      \item \textbf{Prioridad}: media.
    \end{itemize}
  \item \textbf{Historia de usuario 3 (opcional):} como jefe de producción, quiero acceder a predicciones semanales agregadas por línea de producto para facilitar la planificación de turnos y uso de capacidad.
    \begin{itemize}
      \item \textbf{Dificultad}: bajo (3). Se trata de una agregación de las predicciones ya existentes.
      \item \textbf{Complejidad}: bajo (3). Es una funcionalidad de fácil desarrollo.
      \item \textbf{Riesgo}: muy bajo (1). Baja dependencia de factores externos.
      \item \textbf{Story Points} = 3 + 3 + 1 = 8 (valor próximo en la serie de Fibonacci).
      \item \textbf{Prioridad}: baja.
    \end{itemize}
\end{itemize}

\subsection*{6.2. Épica 2: cálculo de márgenes brutos}
\begin{itemize}
  \item \textbf{Historia de usuario 4:} como responsable de control de gestión, quiero estimar los márgenes brutos por canal y producto para evaluar la rentabilidad futura.
    \begin{itemize}
      \item \textbf{Dificultad}: medio (5). Implica cálculo automático y segmentación por canal.
      \item \textbf{Complejidad}: medio (5). Requiere integrar costos y precios de múltiples fuentes.
      \item \textbf{Riesgo}: bajo (3). Riesgo medio por variabilidad en la estructura de costos.
      \item \textbf{Story Points} = 5 + 5 + 3 = 13.
      \item \textbf{Prioridad}: alta.
    \end{itemize}
  \item \textbf{Historia de usuario 5:} como analista financiero, quiero integrar dinámicamente los costos de insumos y logística al cálculo de márgenes para reflejar los cambios del entorno.
    \begin{itemize}
      \item \textbf{Dificultad}: medio (5). Requiere modelar funciones de costo con datos variables.
      \item \textbf{Complejidad}: bajo (3). Debe incluir actualizaciones periódicas y automatización.
      \item \textbf{Riesgo}: muy bajo (1). Los datos suelen estar disponibles en el ERP.
      \item \textbf{Story Points} = 5 + 3 + 1 = 8 (valor más próximo en la serie de Fibonacci).
      \item \textbf{Prioridad}: media.
    \end{itemize}
\end{itemize}

\subsection*{6.3. Épica 3: simulación de escenarios}
\begin{itemize}
  \item \textbf{Historia de usuario 6:} como director comercial, quiero simular escenarios ante cambios en el precio de venta de los productos, para anticipar su impacto económico.
    \begin{itemize}
      \item \textbf{Dificultad}: medio (5). Requiere árboles de decisión y simulación dinámica.
      \item \textbf{Complejidad}: medio (5). Debe considerar elasticidades y sensibilidad de variables.
      \item \textbf{Riesgo}: bajo (3). El modelo puede ser sensible a inputs inciertos.
      \item \textbf{Story Points} = 5 + 5 + 3 = 13.
      \item \textbf{Prioridad}: alta.
    \end{itemize}
  \item \textbf{Historia de usuario 7:} como gerente de planta, quiero modelar la capacidad instalada y la ociosidad según distintos niveles de demanda, para optimizar la producción.
    \begin{itemize}
      \item \textbf{Dificultad}: medio (5). Requiere modelar la planta con datos reales de capacidad.
      \item \textbf{Complejidad}: muy bajo (1). Supone solo parametrizar restricciones operativas.
      \item \textbf{Riesgo}: bajo (1). Baja incertidumbre si se dispone de datos internos.
      \item \textbf{Story Points} = 5 + 1 + 1 = 8 (valor próximo en la serie de Fibonacci).
      \item \textbf{Prioridad}: media.
    \end{itemize}
  \item \textbf{Historia de usuario 8 (opcional):} como analista, quiero simular restricciones operativas (por ejemplo, cortes de gas), para cuantificar su impacto en márgenes.
    \begin{itemize}
      \item \textbf{Dificultad}: bajo (3). Se basa en inputs simples de disponibilidad.
      \item \textbf{Complejidad}: muy bajo (1). La lógica de escenarios es sencilla.
      \item \textbf{Riesgo}: muy bajo (1). No tiene efectos permanentes en el modelo.
      \item \textbf{Story Points} = 3 + 1 + 1 = 5.
      \item \textbf{Prioridad}: baja.
    \end{itemize}
\end{itemize}

\subsection*{6.4. Épica 4: visualización y prueba de concepto (PoC)}
\begin{itemize}
  \item \textbf{Historia de usuario 9:} como usuario final, quiero visualizar los resultados de la predicción y simulación en dashboards interactivos, para tomar decisiones más informadas.
    \begin{itemize}
      \item \textbf{Dificultad}: medio (5). Requiere conectar datos y construir visualizaciones.
      \item \textbf{Complejidad}: bajo (3). Necesita UX clara y efectiva.
      \item \textbf{Riesgo}: bajo (1). Baja probabilidad de error si se usan herramientas estándar.
      \item \textbf{Story Points} = 5 + 3 + 1 = 8.
      \item \textbf{Prioridad}: alta.
    \end{itemize}
  \item \textbf{Historia de usuario 10:} como responsable de planificación, quiero comparar escenarios base y alternativos en un mismo tablero, para evaluar planes de acción.
    \begin{itemize}
      \item \textbf{Dificultad}: bajo (3). Visualización comparativa.
      \item \textbf{Complejidad}: bajo (3). Requiere lógica de selección de escenarios.
      \item \textbf{Riesgo}: muy bajo (1). Los datos ya están procesados.
      \item \textbf{Story Points} = 3 + 3 + 1 = 8 (valor próximo en la serie de Fibonacci).
      \item \textbf{Prioridad}: media.
    \end{itemize}
  \item \textbf{Historia de usuario 11 (opcional):} como responsable financiero, quiero exportar la información a Excel desde Power BI o Streamlit, para facilitar reportes internos.
    \begin{itemize}
      \item \textbf{Dificultad}: bajo (2). Se trata de una funcionalidad estándar.
      \item \textbf{Complejidad}: muy bajo (1). Muy baja.
      \item \textbf{Riesgo}: nulo (0). Prácticamente nulo.
      \item \textbf{Story Points} = 2 + 1 + 0 = 3.
      \item \textbf{Prioridad}: baja.
    \end{itemize}
\end{itemize}

\section{7. Criterios de aceptación de historias de usuario}
\label{sec:criteriosAceptacion}

En esta sección se definen los criterios de aceptación para cada historia de usuario del \textit{backlog}, con el objetivo de establecer condiciones específicas, medibles y verificables que permitan validar la funcionalidad desarrollada. 

Se utiliza el modelo Gherkin, un enfoque ampliamente adoptado en metodologías ágiles para describir escenarios de prueba funcional en lenguaje natural estructurado. Este modelo sigue una sintaxis sencilla basada en las siguientes expresiones clave:

\begin{itemize}
  \item \textbf{Dado (Given)}: establece el contexto inicial o condiciones previas.
  \item \textbf{Cuando (When)}: describe la acción que se va a realizar.
  \item \textbf{Entonces (Then)}: indica el resultado esperado que permite validar el cumplimiento de la funcionalidad.
\end{itemize}

Este formato permite una comprensión clara tanto por parte de usuarios técnicos como no técnicos, facilitando la colaboración y alineamiento entre desarrolladores, analistas y \textit{stakeholders} del proyecto.

\subsection*{7.1. Épica 1: modelado de demanda}

\paragraph{Criterios de aceptación HU1: modelo de predicción de demanda semanal por producto}
\begin{itemize}
  \item \textbf{Criterio 1} \\
  \textbf{Dado} un conjunto de datos históricos de ventas, \\
  \textbf{cuando} se ejecuta el modelo de predicción, \\
  \textbf{entonces} se obtiene una proyección semanal por producto con un error MAPE menor al 30\%.

  \item \textbf{Criterio 2} \\
  \textbf{Dado} un conjunto de productos activos, \\
  \textbf{cuando} el modelo es entrenado, \\
  \textbf{entonces} debe generar una salida estructurada (tabla o archivo) con las predicciones para las siguientes 8 semanas.

  \item \textbf{Criterio 3} \\
  \textbf{Dado} un modelo ya validado, \\
  \textbf{cuando} se realiza una actualización mensual de datos, \\
  \textbf{entonces} el modelo debe actualizar sus parámetros automáticamente y emitir un nuevo pronóstico.
\end{itemize}

\vspace{1em}

\paragraph{Criterios de aceptación HU2: alertas ante desvíos en la demanda real vs. proyectada}
\begin{itemize}
  \item \textbf{Criterio 1} \\
  \textbf{Dado} un desvío superior al 30\% entre la demanda real y la proyectada, \\
  \textbf{cuando} se registre la venta semanal, \\
  \textbf{entonces} el sistema debe generar una alerta visual o por correo electrónico.

  \item \textbf{Criterio 2} \\
  \textbf{Dado} un conjunto de productos monitoreados, \\
  \textbf{cuando} se detecten patrones de desvío sistemático en tres semanas consecutivas, \\
  \textbf{entonces} se debe recomendar una revisión del modelo.
\end{itemize}

\vspace{1em}

\paragraph{Criterios de aceptación HU3 (opcional): predicción agregada por línea de producto}
\begin{itemize}
  \item \textbf{Criterio 1} \\
  \textbf{Dado} un grupo de productos pertenecientes a la misma línea, \\
  \textbf{cuando} se ejecuta la predicción, \\
  \textbf{entonces} se debe mostrar el resultado agregado en una única fila para la línea correspondiente.

  \item \textbf{Criterio 2} \\
  \textbf{Dado} un usuario del área de producción, \\
  \textbf{cuando} consulte la visualización, \\
  \textbf{entonces} podrá seleccionar líneas de producto y visualizar la demanda proyectada semanalmente.
\end{itemize}

\subsection*{7.2. Épica 2: estimación de márgenes brutos por canal}

\paragraph{Criterios de aceptación HU4: cálculo proyectado de márgenes brutos por producto y canal}
\begin{itemize}
  \item \textbf{Criterio 1} \\
  \textbf{Dado} un conjunto de precios de venta estimados y costos variables actualizados, \\
  \textbf{cuando} se ejecute el modelo de márgenes, \\
  \textbf{entonces} debe calcularse automáticamente el margen bruto por producto y canal, expresado en valores absolutos y porcentuales.

  \item \textbf{Criterio 2} \\
  \textbf{Dado} un conjunto de productos con diferentes estructuras de costos, \\
  \textbf{cuando} se visualicen los resultados, \\
  \textbf{entonces} el sistema debe permitir ordenar por margen y detectar productos con contribución negativa.

  \item \textbf{Criterio 3} \\
  \textbf{Dado} un escenario base de estimaciones, \\
  \textbf{cuando} se cambien los precios o costos, \\
  \textbf{entonces} los nuevos márgenes deben actualizarse dinámicamente para reflejar el impacto de la variación.
\end{itemize}

\vspace{1em}

\paragraph{Criterios de aceptación HU5: visualización de márgenes históricos y proyectados}
\begin{itemize}
  \item \textbf{Criterio 1} \\
  \textbf{Dado} un producto seleccionado, \\
  \textbf{cuando} se consulte su historial, \\
  \textbf{entonces} deben mostrarse en una misma visualización sus márgenes brutos históricos y las proyecciones futuras.

  \item \textbf{Criterio 2} \\
  \textbf{Dado} un conjunto de productos agrupados por canal, \\
  \textbf{cuando} se visualicen sus márgenes, \\
  \textbf{entonces} debe identificarse claramente cuáles tienen márgenes por debajo del umbral mínimo definido por la empresa.
\end{itemize}

\subsection*{7.3. Épica 3: simulación de escenarios operativos y financieros}

\paragraph{Criterios de aceptación HU6: construcción de árbol de decisiones para escenarios}
\begin{itemize}
  \item \textbf{Criterio 1} \\
  \textbf{Dado} un conjunto de decisiones estratégicas posibles (e.g., cambio de precio, capacidad instalada, compras), \\
  \textbf{cuando} se cargan como insumos al sistema, \\
  \textbf{entonces} el árbol de decisiones debe construir ramificaciones que representen los distintos caminos posibles y sus resultados económicos esperados.

  \item \textbf{Criterio 2} \\
  \textbf{Dado} un árbol de decisiones generado, \\
  \textbf{cuando} se selecciona una rama o nodo final, \\
  \textbf{entonces} debe mostrarse la utilidad proyectada asociada y el escenario correspondiente de demanda y costos.

  \item \textbf{Criterio 3} \\
  \textbf{Dado} un escenario base, \\
  \textbf{cuando} se simula un cambio en una variable crítica (e.g., precio de insumo), \\
  \textbf{entonces} el árbol debe recalcular los resultados y mostrar las variaciones comparadas al escenario original.
\end{itemize}

\vspace{1em}

\paragraph{Criterios de aceptación HU7: análisis de sensibilidad sobre márgenes}
\begin{itemize}
  \item \textbf{Criterio 1} \\
  \textbf{Dado} un conjunto de variables sensibles (precio, volumen, costos), \\
  \textbf{cuando} se modifique su valor en un rango definido, \\
  \textbf{entonces} el sistema debe mostrar en tiempo real cómo varía el margen bruto por producto y canal.

  \item \textbf{Criterio 2} \\
  \textbf{Dado} un análisis de sensibilidad realizado, \\
  \textbf{cuando} se identifique una variable que impacta fuertemente en el margen, \\
  \textbf{entonces} el sistema debe destacarla como "crítica" mediante una etiqueta visual o comentario.
\end{itemize}

\vspace{1em}

\paragraph{Criterios de aceptación HU8 (opcional): simulación de restricciones operativas}

\begin{itemize}
  \item \textbf{Criterio 1} \\
  \textbf{Dado} un conjunto de restricciones operativas posibles (por ejemplo, cortes de gas o capacidad limitada), \\
  \textbf{cuando} se seleccione una o varias restricciones en el sistema, \\
  \textbf{entonces} debe simularse automáticamente el impacto de estas restricciones sobre el margen bruto total y por canal.

  \item \textbf{Criterio 2} \\
  \textbf{Dado} un escenario con restricciones operativas aplicadas, \\
  \textbf{cuando} se compare con el escenario base sin restricciones, \\
  \textbf{entonces} el sistema debe mostrar claramente las diferencias de rendimiento económico y operatividad.
\end{itemize}

\newpage

\subsection*{7.4. Épica 4: visualización y Prueba de Concepto (PoC)}

\paragraph{Criterios de aceptación HU9: visualización de resultados en dashboards interactivos}
\begin{itemize}
  \item \textbf{Criterio 1} \\
  \textbf{Dado} que el modelo de predicción y simulación ya generó resultados, \\
  \textbf{cuando} se publiquen en una herramienta de visualización, \\
  \textbf{entonces} el usuario debe poder consultar métricas clave por producto y canal de forma dinámica.
  
  \item \textbf{Criterio 2} \\
  \textbf{Dado} un dashboard implementado en Power BI o Streamlit, \\
  \textbf{cuando} el usuario interactúe con filtros de fecha o canal, \\
  \textbf{entonces} los resultados deben actualizarse automáticamente reflejando la selección.
\end{itemize}

\vspace{1em}

\paragraph{Criterios de aceptación HU10: comparación de escenarios}
\begin{itemize}
  \item \textbf{Criterio 1} \\
  \textbf{Dado} un tablero que incluye datos de diferentes escenarios, \\
  \textbf{cuando} el usuario seleccione un escenario base y uno alternativo, \\
  \textbf{entonces} el sistema debe mostrar ambos en paralelo con sus diferencias destacadas.

  \item \textbf{Criterio 2} \\
  \textbf{Dado} que se comparan márgenes y niveles de demanda entre escenarios, \\
  \textbf{cuando} se utilicen herramientas visuales como gráficos de barras o tablas comparativas, \\
  \textbf{entonces} el usuario debe poder identificar fácilmente el impacto de las decisiones evaluadas.
\end{itemize}

\vspace{1em}

\paragraph{Criterios de aceptación HU11 (opcional): exportación a Excel}
\begin{itemize}
  \item \textbf{Criterio 1} \\
  \textbf{Dado} que el usuario final accede al dashboard, \\
  \textbf{cuando} seleccione una tabla de resultados, \\
  \textbf{entonces} debe contar con una opción de exportar a Excel de forma directa y funcional.

  \item \textbf{Criterio 2} \\
  \textbf{Dado} un archivo Excel exportado desde el sistema, \\
  \textbf{cuando} el usuario lo abra, \\
  \textbf{entonces} debe contener los datos visibles en pantalla con las mismas columnas y formatos.
\end{itemize}

\newpage

\section{8. Fases de CRISP-DM}

\subsection*{8.1. Comprensión del negocio}
El objetivo del proyecto es mejorar la planificación financiera en la familia de productos secos (salamines) del ciclo III de la empresa Cagnoli S.A., a través de un sistema inteligente que permita prever la demanda semanal, estimar márgenes brutos por canal y simular escenarios económicos-operativos. El valor agregado del uso de inteligencia artificial (IA) radica en reemplazar métodos heurísticos y subjetivos por herramientas objetivas, basadas en datos históricos y técnicas de \textit{machine learning}. 
El éxito del proyecto será evaluado mediante mejoras en la precisión del pronóstico, la capacidad de análisis de escenarios, y el uso efectivo del sistema por parte de los equipos de planificación.

\subsection*{8.2. Comprensión de los datos}
Los datos provienen de diversas fuentes internas, principalmente del sistema ERP de la compañía (Microsoft Dynamics), planillas de costos, inventarios, pedidos y reportes de ventas. Se cuenta con registros históricos de demanda semanal por producto, precios de venta, estructura de costos variables y datos de producción. La calidad de los datos es adecuada aunque presenta desafíos como series incompletas, variaciones de codificación, series con diferente dimensionalidad o a distinto nivel de detalle (granularidad).

\subsection*{8.3. Preparación de los datos}
Durante esta fase se realizará la limpieza y transformación de los datos, lo cual incluye:
\begin{itemize}
  \item Agregación de la demanda a nivel semanal por producto y canal.
  \item Tratamiento de valores faltantes y atípicos.
  \item Creación de nuevas variables explicativas (festividades, clima, tendencias, promociones).
  \item Normalización de series y construcción de \textit{datasets} de entrenamiento y prueba.
\end{itemize}
La preparación de los datos es crítica para garantizar la calidad del modelo predictivo y la estabilidad de las simulaciones.

\subsection*{8.4. Modelado}
El problema principal es de predicción de series temporales multivariadas y clasificación de escenarios. Se utilizarán modelos estadísticos y de aprendizaje automático como:
\begin{itemize}
  \item SARIMA y Prophet, para capturar estacionalidad y tendencias.
  \item XGBoost y \textit{Random Forest}, para incorporar múltiples variables predictoras y mejorar la robustez frente al ruido.
  \item Árboles de decisión y análisis de sensibilidad, para la simulación de escenarios y evaluación de impacto en márgenes.
\end{itemize}

\subsection*{8.5. Evaluación del modelo}
Los modelos se evaluarán mediante métricas clásicas de rendimiento predictivo, tales como:
\begin{itemize}
  \item MAE (\textit{Mean Absolute Error}) y RMSE (\textit{Root Mean Squared Error}) para pronóstico de demanda.
  \item Error absoluto medio en márgenes estimados para los resultados financieros.
  \item Validación cruzada con períodos de \textit{backtesting} y análisis visual de ajuste de las predicciones.
\end{itemize}

\subsection*{8.6. Despliegue del modelo (opcional)}
El modelo será implementado inicialmente como una prueba de concepto (PoC), mediante herramientas de visualización como Power BI o Streamlit. En esta instancia, el sistema funcionará como un tablero interno para los equipos de planificación, con la capacidad de simular diferentes escenarios y visualizar resultados de predicción. No se prevé integración con el ERP durante esta etapa inicial, aunque sí se contemplan recomendaciones para una futura integración en producción.


\section{9. Desglose del trabajo en tareas}
\label{sec:wbs}

A continuación, se presenta el desglose técnico de las tareas correspondientes a las historias de usuario HU1 a HU4. Se estima una duración entre 2 y 8 horas por tarea, con una prioridad relativa asignada según su impacto en los entregables. Este desglose servirá como base para la planificación en las secciones siguientes (Diagrama de Gantt y Sprints).

Adicionalmente, al inicio se consideran las tareas de planificación del proyecto y selección del director, y finalmente se adicionan las tareas de preparación final de la documentación y práctica previo a la defensa del trabajo final.

\subsection*{9.0. Planificación del proyecto (Previo al desarrollo)}

\begin{table}[H]
\centering
\begin{tabular}{|l|p{6cm}|c|c|}
\hline
\textbf{Historia de usuario} & \textbf{Tarea técnica} & \textbf{Estimación} & \textbf{Prioridad} \\
N/A & Reunión inicial con el cliente y recolección de requerimientos & 3 h & Alta \\
N/A & Lectura de documentos de contexto del negocio & 2 h & Alta \\
N/A & Validación de problemas con stakeholders & 3 h & Alta \\
N/A & Revisión de lineamientos académicos para el plan de proyecto & 3 h & Alta \\
N/A & Búsqueda y contacto de potenciales directores & 2 h & Media \\
N/A & Coordinación de reunión para validación del tema & 3 h & Alta \\
N/A & Redacción inicial del documento de planificación & 6 h & Alta \\
N/A & Elaboración del cronograma preliminar y entregables & 3 h & Alta \\
N/A & Ajuste del documento según feedback & 3 h & Alta \\
N/A & Preparación y entrega formal del plan & 4 h & Alta \\
\hline
\end{tabular}
\caption{Desglose de tareas para la planificación del proyecto.}
\end{table}

\vspace{0.5cm}
\noindent
\textbf{Total estimado parcial:} \textbf{32 horas.}

\subsection*{9.1. Épica 1: modelado de demanda}

\begin{table}[H]
\centering
\begin{tabular}{|l|p{6cm}|c|c|}
\hline
\textbf{Historia de usuario} & \textbf{Tarea técnica} & \textbf{Estimación} & \textbf{Prioridad} \\
\hline
HU1 & Planificación del modelo de demanda & 5 h & Alta \\
HU1 & Selección de herramientas y estrategias de modelado & 5 h & Alta \\
HU1 & Análisis exploratorio de datos históricos & 6 h & Alta \\
HU1 & Visualización de estacionalidad y tendencias & 6 h & Alta \\
HU1 & Detección y tratamiento de outliers y valores nulos & 6 h & Alta \\
HU1 & Estandarización y generación de variables temporales & 6 h & Alta \\
HU1 & Configuración y entrenamiento inicial del modelo SARIMA & 6 h & Alta \\
HU1 & Ajuste fino de parámetros y tuning & 6 h & Alta \\
HU1 & Preparación de inputs y ejecución de Prophet & 6 h & Alta \\
HU1 & Ajuste y evaluación de estacionalidad y feriados & 6 h & Alta \\
HU1 & Optimización de hiperparámetros SARIMA y Prophet & 6 h & Alta \\
HU1 & Evaluación comparativa de precisión con métricas MAE y RMSE & 6 h & Alta \\
HU1 & Validación cruzada temporal con SARIMA y Prophet & 6 h & Alta \\
HU1 & Análisis de errores y comparación final entre modelos & 6 h & Alta \\
HU1 & Documentación técnica del modelo & 8 h & Media \\
HU1 & Pruebas con dataset de validación externo & 5 h & Alta \\
HU1 & Revisión de outputs y trazabilidad de resultados & 5 h & Alta \\
HU1 & Presentación de resultados a usuarios & 8 h & Media \\
\hline
\end{tabular}
\caption{Desglose de tareas para HU1 (Épica 1: modelado de demanda.)}
\end{table}

\begin{table}[H]
\centering
\begin{tabular}{|l|p{6cm}|c|c|}
\hline
\textbf{Historia de usuario} & \textbf{Tarea técnica} & \textbf{Estimación} & \textbf{Prioridad} \\
\hline
HU2 & Definición de variables objetivo y predictores / RF & 5 h & Alta \\
HU2 & Diseño de arquitectura y estrategia de modelado supervisado & 5 h & Alta \\
HU2 & Relevamiento de fuentes externas de datos (clima, estacionalidad) & 5 h & Alta \\
HU2 & Curado y validación de variables externas seleccionadas & 5 h & Alta \\
HU2 & Generación de nuevas variables explicativas & 6 h & Alta \\
HU2 & Evaluación de importancia relativa y reducción de dimensionalidad & 6 h & Alta \\
HU2 & Entrenamiento y validación de XGBoost & 6 h & Alta \\
HU2 & Entrenamiento y validación de Random Forest & 6 h & Alta \\
HU2 & Comparación con SARIMA y Prophet en conjunto de test & 5 h & Alta \\
HU2 & Visualización de resultados y análisis de precisión & 5 h & Alta \\
HU2 & Visualización de errores y residual analysis & 8 h & Media \\
HU2 & Documentación de modelos y variables explicativas & 8 h & Media \\
HU2 & Reunión de feedback con equipo comercial & 8 h & Media \\
\hline
HU3 & Recolección de datos desde ERP de costos variables & 5 h & Alta \\
HU3 & Diseño de estructura de integración y procesamiento de costos & 5 h & Alta \\
HU3 & Modelado del costo variable por producto & 5 h & Alta \\
HU3 & Ajuste para variabilidad mensual, ajustes estacionales y otros costos & 5 h & Alta \\
HU3 & Integración con datos de precios de venta & 8 h & Alta \\
HU3 & Cálculo de márgenes proyectados & 5 h & Alta \\
HU3 & Comparativa de rentabilidad productos por canal & 5 h & Alta \\
HU3 & Análisis de sensibilidad de márgenes & 8 h & Media \\
HU3 & Validación con equipo financiero & 8 h & Alta \\
HU3 & Documentación de supuestos de cálculo & 6 h & Media \\
HU3 & Ajuste final del modelo y revisión & 8 h & Media \\
\hline
\end{tabular}
\caption{Desglose de tareas para HU2 y HU3 (Épica 1: modelado de demanda).}
\end{table}

\vspace{0.2cm}
\noindent
\textbf{Total estimado parcial HU1–HU3:} \textbf{254 horas.}

\subsection*{9.2. Épica 2: estimación de márgenes brutos por canal}

\begin{table}[H]
\centering
\begin{tabular}{|l|p{6cm}|c|c|}
\hline
\textbf{Historia de usuario} & \textbf{Tarea técnica} & \textbf{Estimación} & \textbf{Prioridad} \\
\hline
HU4 & Análisis de escenarios críticos históricos & 8 h & Alta \\
HU4 & Diseño lógico de árbol de decisiones & 8 h & Alta \\
HU4 & Diseño del motor de simulación de escenarios & 6 h & Alta \\
HU4 & Codificación del motor de simulación de escenarios & 6 h & Alta \\
HU4 & Implementación de pruebas con cambios en precios & 5 h & Alta \\
HU4 & Implementación de pruebas con cambios en volúmenes & 5 h & Alta \\
HU4 & Visualización del impacto en márgenes y ventas & 8 h & Media \\
HU4 & Validación de escenarios con dirección & 6 h & Media \\
HU4 & Documentación de escenarios y recomendaciones & 6 h & Media \\
HU4 & Ajustes finales tras feedback del usuario & 6 h & Media \\
\hline
HU5 & Revisión bibliográfica de métodos de simulación & 8 h & Alta \\
HU5 & Diseño de arquitectura funcional del módulo de escenarios & 6 h & Alta \\
HU5 & Diseño de arquitectura técnica del módulo de escenarios & 6 h & Alta \\
HU5 & Lógica de escenarios base: codificación y pruebas & 8 h & Alta \\
HU5 & Lógica de escenarios alternativos: codificación y pruebas & 8 h & Alta \\
HU5 & Integración con outputs de modelos de predicción: análisis de compatibilidad & 5 h & Alta \\
HU5 & Integración con outputs de modelos de predicción: desarrollo e implementación & 6 h & Alta \\
HU5 & Validación preliminar de resultados del simulador & 5 h & Alta \\
HU5 & Validación de casos de uso con usuarios funcionales & 5 h & Alta \\
HU5 & Documentación funcional y técnica del simulador & 8 h & Media \\
HU5 & Revisión y ajuste en base a feedback & 6 h & Media \\
\hline
\end{tabular}
\caption{Desglose de tareas para HU4 a HU5 (Épica 2: estimación de márgenes brutos por canal).}
\end{table}

\vspace{0.2cm}
\noindent
\textbf{Total estimado HU4-HU5:} \textbf{135 horas.}

\subsection*{9.3. Épica 3: simulación de escenarios operativos y financieros}

\begin{table}[H]
\centering
\begin{tabular}{|l|p{6cm}|c|c|}
\hline
\textbf{Historia de usuario} & \textbf{Tarea técnica} & \textbf{Estimación} & \textbf{Prioridad} \\
\hline
HU6 & Recolección de eventos exógenos históricos (clima, insumos) & 5 h & Alta \\
HU6 & Limpieza y estandarización de eventos exógenos & 5 h & Alta \\
HU6 & Modelado de impactos: análisis exploratorio & 6 h & Alta \\
HU6 & Modelado de impactos: implementación & 6 h & Alta \\
HU6 & Diseño de lógica para incorporar shocks al simulador & 6 h & Alta \\
HU6 & Codificación y pruebas de integración de shocks & 6 h & Alta \\
HU6 & Definición de casos de validación & 5 h & Media \\
HU6 & Contraste de resultados del modelo vs. escenarios reales & 5 h & Media \\
HU6 & Documentación de implementación y casos simulados & 8 h & Media \\
\hline
HU7 & Diseño de interfaz para inputs de simulación & 5 h & Alta \\
HU7 & Prototipado y revisión de interfaz con usuarios & 5 h & Alta \\
HU7 & Desarrollo de backend para procesamiento dinámico (parte 1) & 6 h & Alta \\
HU7 & Desarrollo de backend para procesamiento dinámico (parte 2) & 6 h & Alta \\
HU7 & Testeo de inputs y validación de comportamiento & 8 h & Alta \\
HU7 & Manual de uso interactivo del simulador & 6 h & Media \\
HU7 & Recopilación de feedback de usuarios funcionales & 3 h & Media \\
HU7 & Aplicación de mejoras iterativas al simulador & 3 h & Media \\
\hline
HU8 (opcional) & Simulación de restricciones operativas críticas (fase 1) & 5 h & Baja \\
HU8 (opcional) & Simulación de restricciones operativas críticas (fase 2) & 5 h & Baja \\
HU8 (opcional) & Análisis del impacto financiero (datos y supuestos) & 5 h & Baja \\
HU8 (opcional) & Cálculo de impacto en márgenes & 5 h & Baja \\
HU8 (opcional) & Visualización de resultados y diseño de casos & 8 h & Baja \\
HU8 (opcional) & Documentación breve del caso operativo simulado & 4 h & Baja \\
\hline
\end{tabular}
\caption{Desglose de tareas para HU6 a HU8 (Épica 3: simulación de escenarios operativos y financieros).}
\end{table}

\vspace{0.5cm}
\noindent
\textbf{Total estimado HU6-HU8:} \textbf{126 horas.}

\subsection*{9.4. Épica 4: visualización y Prueba de Concepto (PoC)}

\begin{table}[H]
\centering
\begin{tabular}{|l|p{6cm}|c|c|}
\hline
\textbf{Historia de usuario} & \textbf{Tarea técnica} & \textbf{Estimación} & \textbf{Prioridad} \\
\hline
HU9 & Diseño inicial del layout para visualización de predicciones & 5 h & Alta \\
HU9 & Diseño del layout de márgenes y filtros interactivos & 5 h & Alta \\
HU9 & Desarrollo de dashboards: estructura y navegación & 7 h & Alta \\
HU9 & Desarrollo de dashboards: integración con librerías y UX & 7 h & Alta \\
HU9 & Conexión a fuentes de datos internas del modelo & 5 h & Alta \\
HU9 & Validación de conexión y pruebas de actualización automática & 5 h & Alta \\
HU9 & Validación de visualizaciones con usuarios clave & 8 h & Alta \\
HU9 & Ajuste de gráficos y filtros según feedback & 6 h & Media \\
HU9 & Redacción de manual de uso de dashboards & 2 h & Media \\
HU9 & Inclusión de ejemplos y pantallas en la documentación & 2 h & Media \\
\hline
HU10 & Diseño de panel comparativo de escenarios & 8 h & Media \\
HU10 & Lógica de selección de escenarios para comparación & 8 h & Media \\
HU10 & Implementación visual del comparador y controles & 6 h & Media \\
HU10 & Pruebas de casos base vs. alternativos & 8 h & Media \\
HU10 & Documentación técnica del panel comparador & 3 h & Media \\
HU10 & Documentación funcional con ejemplos de uso & 3 h & Media \\
\hline
HU11 (opcional) & Desarrollo de exportación a Excel desde dashboard & 4 h & Baja \\
HU11 (opcional) & Validación de formato y campos exportables & 2 h & Baja \\
HU11 (opcional) & Documentación breve de exportación y soporte & 2 h & Baja \\
\hline
\end{tabular}
\caption{Desglose de tareas para HU9 a HU11 (Épica 4: visualización y Prueba de Concepto (PoC)).}
\end{table}

\vspace{0.5cm}
\noindent
\textbf{Total estimado parcial HU9–HU11:} \textbf{96 horas.}

\subsection*{9.5. Documentación y presentación final}

\begin{table}[H]
\centering
\begin{tabular}{|l|p{6cm}|c|c|}
\hline
\textbf{Historia de usuario} & \textbf{Tarea técnica} & \textbf{Estimación} & \textbf{Prioridad} \\
\hline
N/A & Revisión general del documento final y estructura & 4 h & Alta \\
N/A & Redacción del resumen ejecutivo & 3 h & Alta \\
N/A & Edición final y generación de versión PDF para entrega & 3 h & Alta \\
N/A & Redacción de memoria (parte 1) & 8 h & Alta \\
N/A & Redacción de memoria (parte 2) & 8 h & Alta \\
N/A & Preparación de defensa & 8 h & Alta \\
\hline
\end{tabular}
\caption{Desglose de tareas para la documentación y presentación final.}
\end{table}

\vspace{0.5cm}
\noindent
\textbf{Total estimado parcial:} \textbf{34 horas.}

\noindent
Estimación del costo total del proyecto: \\
Dado que el presente trabajo final requiere una dedicación estimada de 677 horas, y considerando un valor de referencia de USD 40 por hora de ingeniería, el costo total estimado del proyecto asciende a:

\begin{center}
\textbf{677 h $\times$ USD 40/h = USD 27.080}
\end{center}

Esta estimación no representa un costo real incurrido, sino una valorización técnica del esfuerzo y dedicación requeridos para su ejecución, basada en estándares del mercado para profesionales especializados en inteligencia artificial y planificación financiera.

\vspace{0.2cm}
\noindent\dotfill

\vspace{0.5cm}
\noindent
\textbf{Cantidad total de horas: (677 h)}

\vspace{0.2cm}
\noindent\dotfill

\section{10. Diagrama de Gantt}
\label{sec:gantt}

El presente diagrama de Gantt representa de forma visual y cronológica todas las tareas planificadas para el desarrollo del trabajo final. La planificación contempla un total de aproximadamente 677 horas, distribuidas entre tareas técnicas (modelado, simulación, desarrollo de dashboards, validaciones, etc.) y tareas no técnicas (planificación, documentación, redacción de entregables y preparación de la presentación final).

Como primer paso en la planificación, se realizó una distribución detallada de tareas en 15 sprints quincenales. Estos sprints comenzarán luego de la fase inicial de planificación (29/04/2025 al 13/06/2025). Esta división en sprints permitió agrupar tareas según su prioridad, orden lógico y dependencias técnicas, con el fin de asegurar una carga equilibrada de entre 45 y 50 horas por sprint. A partir de esta organización, se construyó el cronograma general representado en el diagrama de Gantt que se muestra a continuación.

El diagrama está estructurado con las tareas organizadas verticalmente y el tiempo representado en el eje horizontal. Cada tarea se asocia a un sprint específico, y se ha utilizado una codificación de colores para distinguir entre tareas técnicas (azul) y no técnicas (naranja). Además, se han configurado las dependencias entre tareas para reflejar adecuadamente la lógica del flujo de trabajo, con el fin de evitar solapamientos innecesarios y permitiendo la trazabilidad del avance del proyecto.

Este cronograma inicia con las actividades de planificación del proyecto (Etapa 0) y culmina con la entrega final y preparación de la defensa, hacia mediados de enero de 2026. En las siguientes figuras se presenta el diagrama de Gantt del proyecto por partes:


\begin{figure}[htpb]
\centering
\fbox{%
  \includegraphics[width=0.95\textwidth]{./Figuras/ProjectGantt_CEIA_UBA_1.png}%
}
\caption{Diagrama de Gantt del Proyecto - Parte 1 de 4.}
\label{fig:diagBloques}
\end{figure}

\vspace{15px}

\begin{figure}[htpb]
\centering
\fbox{%
  \includegraphics[width=0.95\textwidth]{./Figuras/ProjectGantt_CEIA_UBA_2.png}%
}
\caption{Diagrama de Gantt del Proyecto - Parte 2 de 4.}
\label{fig:diagBloques}
\end{figure}

\vspace{15px}

\begin{figure}[htpb]
\centering
\fbox{%
  \includegraphics[width=0.95\textwidth]{./Figuras/ProjectGantt_CEIA_UBA_3.png}%
}
\caption{Diagrama de Gantt del Proyecto - Parte 3 de 4.}
\label{fig:diagBloques}
\end{figure}

\vspace{15px}

\begin{figure}[htpb]
\centering
\fbox{%
  \includegraphics[width=0.95\textwidth]{./Figuras/ProjectGantt_CEIA_UBA_4.png}%
}
\caption{Diagrama de Gantt del Proyecto - Parte 4 de 4.}
\label{fig:diagBloques}
\end{figure}

\vspace{25px}

\section{11. Planificación de Sprints}

El objetivo de esta sección es organizar la ejecución del proyecto en sprints de trabajo con una distribución equilibrada de la carga horaria. La planificación abarca un total de aproximadamente 600 horas, de las cuales entre 480 y 500 horas corresponden a tareas técnicas, y entre 100 y 120 horas a tareas no técnicas, como planificación, redacción de la memoria y preparación de la defensa.

Los sprints definidos tienen una duración de dos semanas cada uno. El primer sprint comienza el 16 de junio de 2025, tras finalizar la etapa de planificación. En total se estructuraron 15 sprints, de los cuales se considera que el último estará destinado exclusivamente a las tareas de cierre del proyecto. La asignación de tareas por sprint se realizó a partir del desglose previo de historias de usuario y tareas técnicas, respetando un límite de entre 45 y 50 horas por sprint.

La tabla incluida en esta sección muestra cada sprint junto con las tareas que lo componen, el número de horas estimadas, el responsable (el autor del proyecto) y el porcentaje de avance previsto o completado. Esta planificación permite establecer prioridades, realizar un seguimiento riguroso del progreso y facilitar instancias de revisión periódica.

\begin{table}[htpb]
\centering
\caption{Planificación inicial (previo al comienzo de sprints).}
\begin{tabularx}{\linewidth}{@{}|l|l|X|c|l|c|@{}}
\hline
\rowcolor[HTML]{C0C0C0}
Sprint & HU o fase & Tarea & Horas & Responsable & \% Completado \\ \hline
N/A & N/A & Reunión inicial con el cliente y recolección de requerimientos & 3 h & Alumno & 100\% \\ \hline
N/A & N/A & Lectura de documentos de contexto del negocio & 2 h & Alumno & 100\% \\ \hline
N/A & N/A & Validación de problemas con stakeholders & 3 h & Alumno & 100\% \\ \hline
N/A & N/A & Revisión de lineamientos académicos para el plan de proyecto & 3 h & Alumno & 100\% \\ \hline
N/A & N/A & Búsqueda y contacto de potenciales directores & 2 h & Alumno & 20\% \\ \hline
N/A & N/A & Coordinación de reunión para validación del tema & 3 h & Alumno & 0\% \\ \hline
N/A & N/A & Redacción inicial del documento de planificación & 6 h & Alumno & 50\% \\ \hline
N/A & N/A & Elaboración del cronograma preliminar y entregables & 3 h & Alumno & 50\% \\ \hline
N/A & N/A & Ajuste del documento según feedback & 3 h & Alumno & 30\% \\ \hline
N/A & N/A & Preparación y entrega formal del plan & 4 h & Alumno & 0\% \\ \hline
\end{tabularx}
\end{table}

\begin{table}[htpb]
\centering
\caption{Planificación de Sprints 1 a 2.}
\begin{tabularx}{\linewidth}{@{}|l|l|X|c|l|c|@{}}
\hline
\rowcolor[HTML]{C0C0C0}
Sprint & HU o fase & Tarea & Horas & Responsable & \% Completado \\ \hline
Sprint 1 & HU1 & Planificación del modelo de demanda & 5 h & Alumno & 0\% \\ \hline
Sprint 1 & HU1 & Selección de herramientas y estrategias de modelado & 5 h & Alumno & 0\% \\ \hline
Sprint 1 & HU1 & Análisis exploratorio de datos históricos & 6 h & Alumno & 0\% \\ \hline
Sprint 1 & HU1 & Visualización de estacionalidad y tendencias & 6 h & Alumno & 0\% \\ \hline
Sprint 1 & HU1 & Detección y tratamiento de outliers y valores nulos & 6 h & Alumno & 0\% \\ \hline
Sprint 1 & HU1 & Estandarización y generación de variables temporales & 6 h & Alumno & 0\% \\ \hline
Sprint 1 & HU1 & Configuración y entrenamiento inicial del modelo SARIMA & 6 h & Alumno & 0\% \\ \hline
Sprint 2 & HU1 & Ajuste fino de parámetros y tuning & 6 h & Alumno & 0\% \\ \hline
Sprint 2 & HU1 & Preparación de inputs y ejecución de Prophet & 6 h & Alumno & 0\% \\ \hline
Sprint 2 & HU1 & Ajuste y evaluación de estacionalidad y feriados & 6 h & Alumno & 0\% \\ \hline
Sprint 2 & HU1 & Optimización de hiperparámetros SARIMA y Prophet & 6 h & Alumno & 0\% \\ \hline
Sprint 2 & HU1 & Evaluación comparativa de precisión con métricas MAE y RMSE & 6 h & Alumno & 0\% \\ \hline
Sprint 2 & HU1 & Validación cruzada temporal con SARIMA y Prophet & 6 h & Alumno & 0\% \\ \hline
Sprint 2 & HU1 & Análisis de errores y comparación final entre modelos & 6 h & Alumno & 0\% \\ \hline
\end{tabularx}
\end{table}


\begin{table}[htpb]
\centering
\caption{Planificación de Sprints 3 a 4.}
\begin{tabularx}{\linewidth}{@{}|l|l|X|c|l|c|@{}}
\hline
\rowcolor[HTML]{C0C0C0}
Sprint & HU o fase & Tarea & Horas & Responsable & \% Completado \\ \hline
Sprint 3 & HU1 & Documentación técnica del modelo & 8 h & Alumno & 0\% \\ \hline
Sprint 3 & HU1 & Pruebas con dataset de validación externo & 5 h & Alumno & 0\% \\ \hline
Sprint 3 & HU1 & Revisión de outputs y trazabilidad de resultados & 5 h & Alumno & 0\% \\ \hline
Sprint 3 & HU1 & Presentación de resultados a usuarios & 8 h & Alumno & 0\% \\ \hline
Sprint 3 & HU1 & Definición de variables objetivo y predictores / RF & 5 h & Alumno & 0\% \\ \hline
Sprint 3 & HU1 & Diseño de arquitectura y estrategia de modelado supervisado & 5 h & Alumno & 0\% \\ \hline
Sprint 3 & HU1 & Relevamiento de fuentes externas de datos (clima, estacionalidad) & 5 h & Alumno & 0\% \\ \hline
Sprint 3 & HU1 & Curado y validación de variables externas seleccionadas & 5 h & Alumno & 0\% \\ \hline
\hline
Sprint 4 & HU2 & Generación de nuevas variables explicativas & 6 h & Alumno & 0\% \\ \hline
Sprint 4 & HU2 & Evaluación de importancia relativa y reducción de dimensionalidad & 6 h & Alumno & 0\% \\ \hline
Sprint 4 & HU2 & Entrenamiento y validación de XGBoost & 6 h & Alumno & 0\% \\ \hline
Sprint 4 & HU2 & Entrenamiento y validación de Random Forest & 6 h & Alumno & 0\% \\ \hline
Sprint 4 & HU2 & Comparación con SARIMA y Prophet en conjunto de test & 5 h & Alumno & 0\% \\ \hline
Sprint 4 & HU2 & Visualización de resultados y análisis de precisión & 5 h & Alumno & 0\% \\ \hline
Sprint 4 & HU2 & Visualización de errores y \textit{residual analysis} & 8 h & Alumno & 0\% \\ \hline
\end{tabularx}
\end{table}

\begin{table}[htpb]
\centering
\caption{Planificación de Sprints 5 a 6.}
\begin{tabularx}{\linewidth}{@{}|l|l|X|c|l|c|@{}}
\hline
\rowcolor[HTML]{C0C0C0}
Sprint & HU o fase & Tarea & Horas & Responsable & \% Completado \\ \hline
Sprint & HU o fase & Tarea & Horas & Responsable & \% Completado \\ \hline
Sprint 5 & HU2 & Documentación de modelos y variables explicativas & 8 h & Alumno & 0\% \\ \hline
Sprint 5 & HU2 & Reunión de feedback con equipo comercial & 8 h & Alumno & 0\% \\ \hline
Sprint 5 & HU3 & Recolección de datos desde ERP de costos variables & 5 h & Alumno & 0\% \\ \hline
Sprint 5 & HU3 & Diseño de estructura de integración y procesamiento de costos & 5 h & Alumno & 0\% \\ \hline
Sprint 5 & HU3 & Modelado del costo variable por producto & 5 h & Alumno & 0\% \\ \hline
Sprint 5 & HU3 & Ajuste para variabilidad mensual, ajustes estacionales y otros costos & 5 h & Alumno & 0\% \\ \hline
Sprint 5 & HU3 & Integración con datos de precios de venta & 8 h & Alumno & 0\% \\ \hline
\hline
Sprint 6 & HU3 & Cálculo de márgenes proyectados & 5 h & Alumno & 0\% \\ \hline
Sprint 6 & HU3 & Comparativa de rentabilidad productos por canal & 5 h & Alumno & 0\% \\ \hline
Sprint 6 & HU3 & Análisis de sensibilidad de márgenes & 8 h & Alumno & 0\% \\ \hline
Sprint 6 & HU3 & Validación con equipo financiero & 8 h & Alumno & 0\% \\ \hline
Sprint 6 & HU3 & Documentación de supuestos de cálculo & 6 h & Alumno & 0\% \\ \hline
Sprint 6 & HU3 & Ajuste final del modelo y revisión & 8 h & Alumno & 0\% \\ \hline
\end{tabularx}
\end{table}

\begin{table}[htpb]
\centering
\caption{Planificación de Sprints 7 a 8.}
\begin{tabularx}{\linewidth}{@{}|l|l|X|c|l|c|@{}}
\hline
\rowcolor[HTML]{C0C0C0}
Sprint & HU o fase & Tarea & Horas & Responsable & \% Completado \\ \hline
Sprint 7 & HU4 & Análisis de escenarios críticos históricos & 8 h & Alumno & 0\% \\ \hline
Sprint 7 & HU4 & Diseño lógico de árbol de decisiones & 8 h & Alumno & 0\% \\ \hline
Sprint 7 & HU4 & Diseño del motor de simulación de escenarios & 6 h & Alumno & 0\% \\ \hline
Sprint 7 & HU4 & Codificación del motor de simulación de escenarios & 6 h & Alumno & 0\% \\ \hline
Sprint 7 & HU4 & Implementación de pruebas con cambios en precios & 5 h & Alumno & 0\% \\ \hline
Sprint 7 & HU4 & Implementación de pruebas con cambios en volúmenes & 5 h & Alumno & 0\% \\ \hline
Sprint 7 & HU4 & Visualización del impacto en márgenes y ventas & 8 h & Alumno & 0\% \\ \hline
\hline
Sprint & HU o fase & Tarea & Horas & Responsable & \% Completado \\ \hline
Sprint 8 & HU4 & Validación de escenarios con dirección & 6 h & Alumno & 0\% \\ \hline
Sprint 8 & HU4 & Documentación de escenarios y recomendaciones & 6 h & Alumno & 0\% \\ \hline
Sprint 8 & HU4 & Ajustes finales tras feedback del usuario & 6 h & Alumno & 0\% \\ \hline
Sprint 8 & HU5 & Revisión bibliográfica de métodos de simulación & 8 h & Alumno & 0\% \\ \hline
Sprint 8 & HU5 & Diseño de arquitectura funcional del módulo de escenarios & 6 h & Alumno & 0\% \\ \hline
Sprint 8 & HU5 & Diseño de arquitectura técnica del módulo de escenarios & 6 h & Alumno & 0\% \\ \hline
Sprint 8 & HU5 & Lógica de escenarios base: codificación y pruebas & 8 h & Alumno & 0\% \\ \hline
\end{tabularx}
\end{table}

\begin{table}[htpb]
\centering
\caption{Planificación de Sprints 9 a 10.}
\begin{tabularx}{\linewidth}{@{}|l|l|X|c|l|c|@{}}
\hline
\rowcolor[HTML]{C0C0C0}
Sprint & HU o fase & Tarea & Horas & Responsable & \% Completado \\ \hline
Sprint 9 & HU5 & Lógica de escenarios alternativos: codificación y pruebas & 8 h & Alumno & 0\% \\ \hline
Sprint 9 & HU5 & Integración con outputs de modelos de predicción: análisis de compatibilidad & 5 h & Alumno & 0\% \\ \hline
Sprint 9 & HU5 & Integración con outputs de modelos de predicción: desarrollo e implementación & 6 h & Alumno & 0\% \\ \hline
Sprint 9 & HU5 & Validación preliminar de resultados del simulador & 5 h & Alumno & 0\% \\ \hline
Sprint 9 & HU5 & Validación de casos de uso con usuarios funcionales & 5 h & Alumno & 0\% \\ \hline
Sprint 9 & HU5 & Documentación funcional y técnica del simulador & 8 h & Alumno & 0\% \\ \hline
Sprint 9 & HU5 & Revisión y ajuste en base a feedback & 6 h & Alumno & 0\% \\ \hline
\hline
Sprint & HU o fase & Tarea & Horas & Responsable & \% Completado \\ \hline
Sprint 10 & HU6 & Recolección de eventos exógenos históricos (clima, insumos) & 5 h & Alumno & 0\% \\ \hline
Sprint 10 & HU6 & Limpieza y estandarización de eventos exógenos & 5 h & Alumno & 0\% \\ \hline
Sprint 10 & HU6 & Modelado de impactos: análisis exploratorio & 6 h & Alumno & 0\% \\ \hline
Sprint 10 & HU6 & Modelado de impactos: implementación & 6 h & Alumno & 0\% \\ \hline
Sprint 10 & HU6 & Diseño de lógica para incorporar shocks al simulador & 6 h & Alumno & 0\% \\ \hline
Sprint 10 & HU6 & Codificación y pruebas de integración de shocks & 6 h & Alumno & 0\% \\ \hline
Sprint 10 & HU6 & Definición de casos de validación & 5 h & Alumno & 0\% \\ \hline
Sprint 10 & HU6 & Contraste de resultados del modelo vs. escenarios reales & 5 h & Alumno & 0\% \\ \hline
\end{tabularx}
\end{table}

\begin{table}[htpb]
\centering
\caption{Planificación de Sprints 11 a 12.}
\begin{tabularx}{\linewidth}{@{}|l|l|X|c|l|c|@{}}
\hline
\rowcolor[HTML]{C0C0C0}
Sprint & HU o fase & Tarea & Horas & Responsable & \% Completado \\ \hline
Sprint 11 & HU6 & Documentación de implementación y casos simulados & 8 h & Alumno & 0\% \\ \hline
Sprint 11 & HU7 & Diseño de interfaz para inputs de simulación & 5 h & Alumno & 0\% \\ \hline
Sprint 11 & HU7 & Prototipado y revisión de interfaz con usuarios & 5 h & Alumno & 0\% \\ \hline
Sprint 11 & HU7 & Desarrollo de backend para procesamiento dinámico (parte 1) & 6 h & Alumno & 0\% \\ \hline
Sprint 11 & HU7 & Desarrollo de backend para procesamiento dinámico (parte 2) & 6 h & Alumno & 0\% \\ \hline
Sprint 11 & HU7 & Testeo de inputs y validación de comportamiento & 8 h & Alumno & 0\% \\ \hline
Sprint 11 & HU7 & Manual de uso interactivo del simulador & 6 h & Alumno & 0\% \\ \hline
Sprint 11 & HU7 & Recopilación de feedback de usuarios funcionales & 3 h & Alumno & 0\% \\ \hline
Sprint 11 & HU7 & Aplicación de mejoras iterativas al simulador & 3 h & Alumno & 0\% \\ \hline
\hline
Sprint 12 & HU8 & Simulación de restricciones operativas críticas (fase 1) & 5 h & Alumno & 0\% \\ \hline
Sprint 12 & HU8 & Simulación de restricciones operativas críticas (fase 2) & 5 h & Alumno & 0\% \\ \hline
Sprint 12 & HU8 & Análisis del impacto financiero (datos y supuestos) & 5 h & Alumno & 0\% \\ \hline
Sprint 12 & HU8 & Cálculo de impacto en márgenes & 5 h & Alumno & 0\% \\ \hline
Sprint 12 & HU8 & Visualización de resultados y diseño de casos & 8 h & Alumno & 0\% \\ \hline
Sprint 12 & HU8 & Documentación breve del caso operativo simulado & 4 h & Alumno & 0\% \\ \hline
Sprint 12 & HU9 & Diseño inicial del layout para visualización de predicciones & 5 h & Alumno & 0\% \\ 
Sprint 12 & HU9 & Diseño del layout de márgenes y filtros interactivos & 5 h & Alumno & 0\% \\ \hline
\end{tabularx}
\end{table}

\begin{table}[htpb]
\centering
\caption{Planificación de Sprints 13 a 14.}
\begin{tabularx}{\linewidth}{@{}|l|l|X|c|l|c|@{}}
\hline
\rowcolor[HTML]{C0C0C0}
Sprint & HU o fase & Tarea & Horas & Responsable & \% Completado \\ \hline
Sprint 13 & HU9 & Desarrollo de dashboards: estructura y navegación & 7 h & Alumno & 0\% \\ \hline
Sprint 13 & HU9 & Desarrollo de dashboards: integración con librerías y UX & 7 h & Alumno & 0\% \\ \hline
Sprint 13 & HU9 & Conexión a fuentes de datos internas del modelo & 5 h & Alumno & 0\% \\ \hline
Sprint 13 & HU9 & Validación de conexión y pruebas de actualización automática & 5 h & Alumno & 0\% \\ \hline
Sprint 13 & HU9 & Validación de visualizaciones con usuarios clave & 8 h & Alumno & 0\% \\ \hline
Sprint 13 & HU9 & Ajuste de gráficos y filtros según feedback & 6 h & Alumno & 0\% \\ \hline
Sprint 13 & HU9 & Redacción de manual de uso de dashboards & 2 h & Alumno & 0\% \\ \hline
Sprint 13 & HU9 & Inclusión de ejemplos y pantallas en la documentación & 2 h & Alumno & 0\% \\ \hline
\hline
Sprint 14 & HU10 & Diseño de panel comparativo de escenarios & 8 h & Alumno & 0\% \\ \hline
Sprint 14 & HU10 & Lógica de selección de escenarios para comparación & 8 h & Alumno & 0\% \\ \hline
Sprint 14 & HU10 & Implementación visual del comparador y controles & 6 h & Alumno & 0\% \\ \hline
Sprint 14 & HU10 & Pruebas de casos base vs. alternativos & 8 h & Alumno & 0\% \\ \hline
Sprint 14 & HU10 & Documentación técnica del panel comparador & 3 h & Alumno & 0\% \\ \hline
Sprint 14 & HU10 & Documentación funcional con ejemplos de uso & 3 h & Alumno & 0\% \\ \hline
Sprint 14 & HU11 & Desarrollo de exportación a Excel desde dashboard & 4 h & Alumno & 0\% \\ \hline
Sprint 14 & HU11 & Validación de formato y campos exportables & 2 h & Alumno & 0\% \\ \hline
Sprint 14 & HU11 & Documentación breve de exportación y soporte & 2 h & Alumno & 0\% \\ \hline
\end{tabularx}
\end{table}

\begin{table}[htpb]
\centering
\caption{Planificación de Sprint 15.}
\begin{tabularx}{\linewidth}{@{}|l|l|X|c|l|c|@{}}
\hline
\rowcolor[HTML]{C0C0C0}
Sprint & HU o fase & Tarea & Horas & Responsable & \% Completado \\ \hline
Sprint 15 & N/A & Revisión general del documento final y estructura & 4 h & Alumno & 0\% \\ \hline
Sprint 15 & N/A & Redacción del resumen ejecutivo & 3 h & Alumno & 0\% \\ \hline
Sprint 15 & N/A & Edición final y generación de versión PDF para entrega & 3 h & Alumno & 0\% \\ \hline
Sprint 15 & N/A & Redacción de memoria (parte 1) & 8 h & Alumno & 0\% \\ \hline
Sprint 15 & N/A & Redacción de memoria (parte 2) & 8 h & Alumno & 0\% \\ \hline
Sprint 15 & N/A & Preparación de defensa & 8 h & Alumno & 0\% \\ \hline
\end{tabularx}
\end{table}


\section{12. Normativa y cumplimiento de datos (gobernanza)}

El presente proyecto se apoya en el uso de datos operativos, comerciales y financieros provenientes de una empresa privada del sector alimenticio, específicamente vinculada a la producción y comercialización de fiambres y chacinados. Estos datos incluyen volúmenes de producción, series históricas de demanda, precios, márgenes, eventos exógenos y estructuras de costos, entre otros elementos. Dado el carácter sensible y estratégico de parte de esta información, resulta clave garantizar el cumplimiento normativo y el respeto de los principios de gobernanza de datos a lo largo de todo el desarrollo del proyecto.

\subsection*{12.1. Normativas aplicables}

Los datos utilizados no incluyen información personal de clientes, empleados ni terceros identificables, por lo tanto no se encuentran comprendidos bajo la Ley Nacional 25.326 de Protección de Datos Personales (Argentina), ni bajo el Reglamento General de Protección de Datos de la Unión Europea (GDPR), ni normativas específicas como HIPAA (para el ámbito de la salud). Sin embargo, al trabajar con datos de carácter confidencial, operativos y estratégicos de una empresa, deben observarse los siguientes principios normativos y éticos:

\begin{itemize}
  \item Confidencialidad: todo el procesamiento y análisis de datos se realiza bajo un acuerdo explícito con la empresa, garantizando la confidencialidad de la información.
  \item Licitud y finalidad: los datos son utilizados exclusivamente para los fines establecidos en el marco de este proyecto académico y no serán reutilizados con otros propósitos sin el consentimiento explícito de la empresa.
  \item Minimización de datos: se evita recolectar información innecesaria, y se emplea únicamente aquella requerida para alcanzar los objetivos del proyecto.
  \item Transparencia: se documenta el origen, tratamiento y aplicación de cada conjunto de datos, lo que facilita auditorías internas o académicas.
\end{itemize}

\subsection*{12.2. Condiciones de uso de los datos}

El acceso a los datos fue autorizado por la organización en el marco de un proyecto colaborativo, donde el autor del presente trabajo se desempeña como consultor externo. Esta relación garantiza un marco de trabajo ético y profesional, bajo los siguientes términos:

\begin{itemize}
  \item Acceso controlado: los datos no fueron compartidos públicamente ni almacenados en plataformas abiertas. El almacenamiento y procesamiento se realizará en entornos locales o corporativos seguros.
  \item No publicación sin anonimización: en caso de presentaciones académicas o publicaciones, cualquier dato será anonimizado o presentado de forma agregada, de modo de preservar la confidencialidad del negocio.
  \item Sin involucramiento de datos personales: no se procesan nombres, identificadores personales, correos electrónicos ni ningún tipo de dato sensible sobre personas físicas.
  \item Consentimiento institucional: el uso de los datos fue aprobado por los responsables de la empresa con el objetivo de contribuir al desarrollo del sistema de planificación y control.
\end{itemize}

\subsection*{12.3. Evaluación de restricciones y viabilidad}

Desde el punto de vista legal y técnico, el uso de los datos es viable, siempre que se mantengan los criterios establecidos anteriormente. No existen restricciones contractuales, regulatorias o legales explícitas que impidan su tratamiento para fines académicos internos. Tampoco se requiere consentimiento de terceros, dado que no se trata de información asociada a individuos.

No obstante, se reconoce que parte de la información tiene valor estratégico para la empresa (e.g., márgenes, simulaciones de precios, estructura de costos), por lo cual se reafirma el compromiso de mantener la confidencialidad y de evitar cualquier exposición no autorizada de los resultados.

\subsection*{12.4. Gobernanza de datos}

El proyecto promueve buenas prácticas de gobernanza de datos. Para lograr esto se asegura lo siguiente:

\begin{itemize}
  \item Registro detallado de las fuentes de datos utilizadas, sus características, y las transformaciones aplicadas.
  \item Reproducibilidad del análisis: todos los procesos pueden ser auditados internamente por la empresa.
  \item Integración con políticas internas de seguridad: el diseño del modelo considera que, en etapas futuras, podría integrarse con herramientas empresariales como Power BI, Streamlit o el ERP Microsoft Dynamics, bajo protocolos seguros.
  \item Documentación y versionado: cada etapa de desarrollo ha sido documentada para facilitar el traspaso y sostenibilidad del sistema en el tiempo.
\end{itemize}

\subsection*{Conclusión}

El presente proyecto cumple con los lineamientos legales, éticos y técnicos en materia de uso de datos. Si bien no se manejan datos personales ni se viola ninguna normativa de protección de datos vigente, se aplican estándares de gobernanza orientados a garantizar la transparencia, trazabilidad y confidencialidad de la información utilizada. Esto no solo resguarda a la organización involucrada, sino que también fortalece la robustez y profesionalismo del trabajo realizado.


\section{13. Gestión de riesgos}
\label{sec:riesgos}

En esta sección se identifican y analizan los principales riesgos asociados al desarrollo del presente proyecto. Se evalúan tanto su severidad (S) como su probabilidad de ocurrencia (O), utilizando una escala del 1 al 10, y se calcula el RPN (Risk Priority Number) como producto de ambas variables. A continuación, se presentan las justificaciones, la tabla de riesgos y los planes de mitigación correspondientes.

a) Identificación de los riesgos (al menos cinco) y estimación de sus consecuencias:


Riesgo 1: retrasos en la entrega de datos por parte del área financiera o comercial.
\begin{itemize}
	\item Severidad (S): 8. Los datos son necesarios para alimentar los modelos de márgenes y simulación, afectando el cronograma completo si no están disponibles.
	\item Probabilidad de ocurrencia (O): 5. Aunque existe buena predisposición, pueden ocurrir demoras por prioridades internas o validaciones.\\
\end{itemize}   

Riesgo 2: incompatibilidad entre formatos de datos extraídos del ERP y lo requerido por el modelo.
\begin{itemize}
	\item Severidad (S): 6. Puede requerir tiempo adicional en preprocesamiento y scripts de transformación.
	\item Ocurrencia (O): 6. Ya se observaron casos de formatos inconsistentes entre áreas.
\end{itemize}

Riesgo 3: incertidumbre en el comportamiento de modelos de IA ante cambios estacionales abruptos.
\begin{itemize}
	\item Severidad (S):  7. Podría reducir la confiabilidad del modelo de predicción de demanda.
	\item Ocurrencia (O): 6. Históricamente hubo saltos estacionales no triviales en la demanda.
\end{itemize}

Riesgo 4: falta de disponibilidad del consultor (responsable del proyecto) por compromisos laborales.
\begin{itemize}
	\item Severidad (S):  8. El proyecto depende fuertemente del responsable.
	\item Ocurrencia (O): 4. Aunque el calendario está planificado, existe carga de trabajo simultánea.
\end{itemize}

Riesgo 5: dificultad para validar escenarios simulados con usuarios clave por falta de participación.
\begin{itemize}
	\item Severidad (S):  6. Los escenarios pueden no ser creíbles ni útiles si no se validan con expertos.
	\item Ocurrencia (O): 5. Ya se experimentaron demoras en validaciones anteriores.
\end{itemize}

Riesgo 6: fallos técnicos en ejecución de modelos (capacidad limitada en entorno local).
\begin{itemize}
	\item Severidad (S): 9. Puede frenar completamente tareas de entrenamiento o simulación.
	\item Ocurrencia (O): 4. El entorno local tiene recursos limitados, pero se prevé migración a la nube si es necesario.
\end{itemize}

Riesgo 7: dificultad para integrar márgenes y demanda en simulación en el tiempo proyectado.
\begin{itemize}
	\item Severidad (S):  7. Puede dejar incompleta una de las funcionalidades principales del trabajo.
	\item Ocurrencia (O): 5. La integración requiere diseño modular y sincronización entre subsistemas.
\end{itemize}

Riesgo 8: rechazo del modelo final por usuarios por falta de explicabilidad.
\begin{itemize}
	\item Severidad (S):  6. Disminuye su aplicabilidad y genera resistencia a su uso.
	\item Ocurrencia (O): 3. Dependerá de cómo se comuniquen los resultados y del nivel de documentación.
\end{itemize}

Riesgo 9: pérdida de avances por falta de backups.
\begin{itemize}
	\item Severidad (S):  7. Implicaría rehacer parte del trabajo.
	\item Ocurrencia (O): 2. Se hacen respaldos, pero podrían no estar actualizados.
\end{itemize}

Riesgo 10: cambios en prioridades internas de la empresa que posterguen validaciones.
\begin{itemize}
	\item Severidad (S):  6. Afecta entregables planificados.
	\item Ocurrencia (O): 4. Aunque el proyecto está acordado, pueden cambiar los tiempos según la coyuntura.
\end{itemize}

\newpage
b) Tabla de gestión de riesgos:      (El RPN se calcula como RPN=SxO)


\begin{table}[htpb]
\centering
\rowcolors{2}{white}{gray!10}
\begin{tabular}{|p{5cm}|c|c|c|c|c|c|}
\hline
\rowcolor[HTML]{C0C0C0}
\textbf{Riesgo} & \textbf{S} & \textbf{O} & \textbf{RPN} & \textbf{S*} & \textbf{O*} & \textbf{RPN*} \\
\hline
1. Retraso en entrega de datos & 8 & 5 & \cellcolor{red!30}40 & 6 & 3 & \cellcolor{yellow!30}18 \\
2. Incompatibilidad de formatos de datos & 6 & 6 & \cellcolor{red!30}36 & 4 & 4 & \cellcolor{yellow!30}16 \\
3. Modelos no robustos a cambios estacionales & 7 & 6 & \cellcolor{red!30}42 & 5 & 4 & \cellcolor{yellow!30}20 \\
4. Falta de disponibilidad del responsable & 8 & 4 & \cellcolor{yellow!30}32 & - & - & - \\
5. Escasa participación en validación de escenarios & 6 & 5 & \cellcolor{yellow!30}30 & - & - & - \\
6. Fallos técnicos en entorno local & 9 & 4 & \cellcolor{red!30}36 & 5 & 3 & \cellcolor{yellow!30}15 \\
7. Problemas en integración márgenes-demanda & 7 & 5 & \cellcolor{red!30}35 & 5 & 3 & \cellcolor{yellow!30}15 \\
8. Rechazo por falta de explicabilidad & 6 & 3 & \cellcolor{yellow!30}18 & - & - & - \\
9. Pérdida de avances sin backup & 7 & 2 & \cellcolor{yellow!30}14 & - & - & - \\
10. Cambio de prioridades internas & 6 & 4 & \cellcolor{yellow!30}24 & - & - & - \\
\hline
\end{tabular}
\caption{Tabla de gestión de riesgos (se considera crítica toda RPN mayor o igual a 35).}
\end{table}

Criterio adoptado: 

Se tomarán medidas de mitigación en los riesgos cuyos números de RPN sean mayores o iguales a 35

Nota: los valores marcados con (*) en la tabla corresponden luego de haber aplicado la mitigación.

c) Plan de mitigación de los riesgos que originalmente excedían el RPN máximo establecido:

Dado que varios de los riesgos identificados comparten factores de origen comunes, (como la dependencia de datos externos, limitaciones técnicas y plazos acotados), se adopta una estrategia general de mitigación centrada en: (I) validación temprana, (II) automatización de tareas críticas, (III) coordinación proactiva con áreas responsables y (IV) planificación de escenarios alternativos. Esta estrategia se adapta específicamente a cada riesgo para reducir su severidad y/o probabilidad de ocurrencia.

Riesgo 1: retrasos en la entrega de datos por parte del área financiera o comercial.

Plan de mitigación específico: coordinar reuniones tempranas para alinear expectativas de entrega, generar alertas automáticas ante demoras y contar con fuentes de datos de respaldo.
\begin{itemize}
	\item Severidad (S):  6. Se reduce  mediante una mayor capacidad de reacción frente a demoras previstas.
	\item Ocurrencia (O): 3. Se reduce al establecer canales de comunicación y fechas acordadas formalmente.
\end{itemize}

Riesgo 2: incompatibilidad entre formatos de datos extraídos del ERP y lo requerido por el modelo

Plan de mitigación específico: desarrollar scripts de conversión automatizados y definir plantillas estándar para el intercambio de datos desde el inicio del proyecto.
\begin{itemize}
	\item Severidad (S*): 4. Se reduce al evitar errores manuales y acelerar la adaptación de datos.
	\item Probabilidad de ocurrencia (O*): 4. Disminuye con la validación temprana de estructuras de datos.
\end{itemize}

Riesgo 3: incertidumbre en el comportamiento de modelos de IA ante cambios estacionales abruptos

Plan de mitigación específico: aplicar validación cruzada temporal, incorporar variables exógenas (clima, calendario), y mantener actualizados los modelos con nuevos datos.
\begin{itemize}
	\item Severidad (S*): 5. Disminuye al anticipar escenarios adversos.
	\item Probabilidad de ocurrencia (O*): 4. Se reduce mediante ajustes periódicos de los modelos.
\end{itemize}

Riesgo 6: fallos técnicos en ejecución de modelos (capacidad limitada en entorno local)

Plan de mitigación específico: preparar entorno de trabajo en la nube como alternativa, reducir el volumen de datos en pruebas intermedias y optimizar código.
\begin{itemize}
	\item Severidad (S*): 7. Baja al contar con opciones escalables y respaldo técnico.
	\item Probabilidad de ocurrencia (O*): 2. Se reduce con planificación anticipada de infraestructura.
\end{itemize}

Riesgo 7: dificultad para integrar márgenes y demanda en simulación en el tiempo proyectado

Plan de mitigación específico: modularizar la simulación por componentes (demanda y márgenes por separado), establecer entregables parciales y validar lógica por fases.
\begin{itemize}
	\item Severidad (S*): 5. Se reduce por una mejor organización del desarrollo.
	\item Probabilidad de ocurrencia (O*): 3. Se mitiga al aplicar integración progresiva de componentes.
\end{itemize}

\newpage

\section{14. Sprint Review}
\label{sec:sprint_review}

La revisión de sprint es una instancia clave en la metodología ágil adoptada, ya que permite validar el cumplimiento parcial del backlog y anticipar posibles desvíos en el desarrollo. A continuación, se detallan las funcionalidades más relevantes de cinco historias de usuario seleccionadas del backlog. Para cada una se identifican las tareas técnicas asociadas, el entregable esperado, los criterios de aceptación, y los principales riesgos u observaciones.


\begin{table}[htbp]
\small
\centering
\caption{Evaluación proyectada de funcionalidades clave del backlog.}
\begin{tabularx}{\linewidth}{|c|c|X|X|X|}
\hline
\rowcolor[HTML]{C0C0C0}
\textbf{HU seleccionada} & \textbf{Tareas asociadas} & \textbf{Entregable esperado} & \textbf{¿Cómo sabrás que está cumplida?} & \textbf{Observaciones o riesgos} \\ \hline

HU2 & Tarea 1 a 4 & Set validado de predictores y datos externos & Variables alineadas con definición y sin problemas de carga & Requiere disponibilidad de fuentes externas \\ \hline
HU2 & Tarea 5 a 9 & Modelos XGBoost y RF entrenados & Métricas de evaluación aceptables en conjunto de test & Posible sobreajuste o sesgo en los datos \\ \hline
HU3 & Tarea 1 a 5 & Módulo de márgenes con datos reales & Cálculo reproducible de márgenes unitarios y totales & Validación con equipo financiero es crítica \\ \hline
HU4 & Tarea 1 a 6 & Simulador de escenarios comerciales & Permite simular cambios de precios y volúmenes con impacto visual & Riesgo de inconsistencias en lógica y validación \\ \hline
HU5 & Tarea 1 a 5 & Módulo de integración de simulación con predicciones & Panel funcional y verificado con usuarios & Riesgo técnico en compatibilidad de outputs entre modelos \\ \hline
HU9 & Tarea 1 a 5 & Dashboard principal integrado & Visualización funcional de métricas y filtros clave & Feedback del usuario puede requerir rediseños \\ \hline
\end{tabularx}
\end{table}

\section{15. Sprint Retrospective}    
\label{sec:sprint_retro}

La retrospectiva de sprint permite reflexionar y establecer mejoras sobre la dinámica de trabajo durante el desarrollo del proyecto. A partir de esta práctica, se identifican acciones para reforzar lo que funcionó, mejorar lo que no, y planificar nuevas formas de trabajo.

Se presenta a continuación una tabla basada en la herramienta “Estrella de la Retrospectiva”, con foco en tres sprints técnicos clave (1, 2 y 4) y uno no técnico (15), seleccionados por su importancia en la etapa inicial, de entrenamiento de modelos y de cierre del proyecto.

\begin{table}[H]
\centering
\caption{Reflexión por Sprint – Estrella de la Retrospectiva.}
\resizebox{\textwidth}{!}{
\begin{tabular}{|l|p{3.2cm}|p{3cm}|p{3cm}|p{3cm}|p{3cm}|}
\hline
\rowcolor[HTML]{C0C0C0}
\textbf{Sprint} & \textbf{¿Qué hacer más?} & \textbf{¿Qué hacer menos?} & \textbf{¿Qué mantener?} & \textbf{¿Qué empezar a hacer?} & \textbf{¿Qué dejar de hacer?} \\ \hline

Técnico – 1 & Validaciones frecuentes del modelo con pruebas intermedias & Correcciones sin documentar & Visualización clara de patrones temporales & Registrar cambios en estructura de variables & Iterar sin guardar versiones intermedias \\ \hline

Técnico – 2 & Comparaciones explícitas entre modelos con métricas claras & Repeticiones sin control de versiones & Curvas de validación temporal & Documentar supuestos del modelo & Entrenamientos sin control de hiperparámetros \\ \hline

Técnico – 4 & Testeo de modelos con inputs alternativos & Ajustes de parámetros sin justificación & Matriz comparativa de resultados entre modelos & Anotar configuraciones óptimas & Cambiar parámetros sin guardar configuración \\ \hline

No técnico – 15 & Consolidación visual de la memoria & Cambios innecesarios en versiones finales & Estructura por bloques temáticos & Separar visuales en anexos & Forzar gráficos que no aportan claridad \\ \hline

\end{tabular}
}
\end{table}

\end{document}